\chapter{Introduction}\label{chap_introduction}
\setlength{\parskip}{12pt}

The Common Community Physics Package (CCPP) is designed to facilitate the implementation of physics innovations in state of the art atmospheric models and the transition of physics packages from one model to another. The CCPP consists of two separate software packages, the pool of CCPP-compliant physics schemes (\execout{ccpp-physics}) and the framework (driver) that connects the physics schemes with a host model (\execout{ccpp-framework}).

The connection between the host model and the physics schemes through the CCPP framework is realized with caps on both sides as illustrated in Fig.~\ref{fig_ccpp_design_with_ccpp_prebuild} in Chapter~\ref{chap_hostmodel}. While the caps to the individual physics schemes are auto-generated, the cap that connects the framework (Physics Driver) to the host model must be created manually.

This document serves two purposes, namely to describe how to write a CCPP-compliant physics scheme and add it to the pool of CCPP physics schemes (chapter~\ref{chap_schemes}), and to explain in detail the process of connecting an atmospheric model (host model) with the CCPP (chapter~\ref{chap_hostmodel}). For further information and an example for integrating CCPP with a host model, the reader is referred to the GMTB Single Column Model (SCM) Technical Guide v1.0 available at {\red\url{MISSING}}.

At the time of writing, CCPP is integrated and tested with the GMTB Single Column Model (SCM) and the GFDL Finite-Volume Cubed-Sphere Model (FV3). While the code governance for the host models lies with the respective organizations, the pool of CCPP physics and the CCPP infrastructure are managed by GMTB and governed by ... {\red MISSING, LIGIA PLEASE ADD INFORMATION HERE}. The GMTB welcomes contributions to CCPP that can be made in form of git pull requests to the respective development repositories. For further information, see the Developer Information for GMTB CCPP at \url{https://dtcenter.org/gmtb/users/ccpp/developers/index.php}.
